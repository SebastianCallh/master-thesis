% Created 2019-01-10 tor 12:10
% Intended LaTeX compiler: pdflatex
\documentclass[11pt]{article}
\usepackage[utf8]{inputenc}
\usepackage[T1]{fontenc}
\usepackage{graphicx}
\usepackage{grffile}
\usepackage{longtable}
\usepackage{wrapfig}
\usepackage{rotating}
\usepackage[normalem]{ulem}
\usepackage{amsmath}
\usepackage{textcomp}
\usepackage{amssymb}
\usepackage{capt-of}
\usepackage{hyperref}
\author{Sebastian}
\date{\today}
\title{}
\hypersetup{
 pdfauthor={Sebastian},
 pdftitle={},
 pdfkeywords={},
 pdfsubject={},
 pdfcreator={Emacs 26.1 (Org mode 9.1.9)}, 
 pdflang={English}}
\begin{document}

\tableofcontents

\section{Master Thesis}
\label{sec:orgab6ece1}

\subsection{Startup meeting with Heintz}
\label{sec:orge042a7d}
\textit{<2019-01-07 mån>   }   
\subsubsection{Initial project- Arrival time prediction with a motion pattern model using GPs}
\label{sec:orgbba6c41}
\begin{itemize}
\item A working implementation exists from this years TDDE19
\item Compare with vector field (position -> velocity), LSTM (from TDDE19)
\end{itemize}

\subsubsection{Areas of improvement from TDDE19}
\label{sec:org3d4f12b}
\begin{itemize}
\item Sparse GPs
\item Synchronisation. Interpolate posterior mean functions, cluster.
\item Data augmentation
\item Start and stop boundaries
\item Multiple stop prediction
\item Time table feature
\end{itemize}

\subsubsection{Notes}
\label{sec:orgbc6a324}
Heintz had some ideas on interesting problems.

\begin{enumerate}
\item Superimposing effects
\label{sec:org2a87f74}
The idea is in isolation train a nominal trajectory, one for rush
hour, on for normal traffics, etc. And in some way superimpose
the effects, letting us ask questions like ``what if we drive
this way instead?'' or  ``what if it was winter?''.

\item Knowledge extraction
\label{sec:orgd70e2bf}
Predicting arrival time is fine, but not novel
enough. Fortunately, the GP model offers plenty of insight in the
data generating process. 

\begin{enumerate}
\item Worst/best scenarios
\label{sec:org484a53b}
If the model was extended to work for
multiple future stops it could give an estimate of the earliest
known arrival time, and the latest known. 

\item Pattern extraction
\label{sec:orgc9735b7}
It would be interesting to extract patterns from the
data, both spatially and temporally. For instance: stop lights, 
traffic jams, average speed (?), speed change. During what circumstances
does the predictions have highest variance?
\end{enumerate}
\end{enumerate}

\subsubsection{Post meeting talk}
\label{sec:org9c3f87a}
Some ideas that felt reasonable after talking to Tiger after the
meeting with Heintz.
\begin{itemize}
\item Extract events using kernels/convolution over motion patterns
\item Extending the application to multiple future stops
\item Comparing predictive distribution to the actual arrival times,
and the predictions of LSTM/vector field model
\end{itemize}

\subsection{Literature}
\label{sec:orgdf29339}
\subsubsection{Related work}
\label{sec:orgd458d71}

\subsection{Things to do}
\label{sec:org3f5ea9e}
\subsubsection{{\bfseries\sffamily TODO} Download pdfs of all papers and put them in shared documents}
\label{sec:orgd0757b2}
\subsubsection{{\bfseries\sffamily DONE} Mail Heintz and Tiger about TP meeting jan 18}
\label{sec:orga93cec8}

\subsubsection{{\bfseries\sffamily DONE} Figure out what the heck to do}
\label{sec:orgfccb347}
Talk to Mattias about this. Motivate what value this brings. What
is novel about it?

\subsubsection{{\bfseries\sffamily TODO} Conduct thorough literature study}
\label{sec:org53bdf36}
\begin{itemize}
\item[{$\square$}] On the domain, motivate why the problem is interesting \hyperref[tiger-question-1]{question}
This is currently motivated with a non-peer reviewed article on the importance on

\item[{$\square$}] On related work, what has been done previously
See individual \ref{sub-problems}

\item[{$\square$}] On the chosen solution, motivate why this is valid
\end{itemize}
\begin{enumerate}
\item Trajectory model
\label{sec:orge69d015}
GPR successfully used for trajectories \cite{Kim2011Nov}

\item Synchronisation
\label{sec:orgb234e6f}
\item Similarity metric
\label{sec:org82c3906}
\item Clustering algorithm
\label{sec:orgea0a93f}
\item Regression model
\label{sec:orgcab3a8c}
\cite{Rasmussen-Williams-2006} claim GPs ``a serious competitor for real supervised learning applications''

\begin{itemize}
\item[{$\boxtimes$}] On the chosen solution, show how this improves on previous work
Some ideas are: Explainability, find good or bad
patterns/events, speed changes, stops
Compared to LSTMs it comes with a posterior
Outliers can be detected
Best/worst case scenarios
\end{itemize}
\end{enumerate}

\subsubsection{Write down potential solutions on different sub-problems}
\label{sec:orgd2a7505}
\label{sub-problems}
\begin{enumerate}
\item {\bfseries\sffamily TODO} Comparing trajectories
\label{sec:orgf941a30}
A distance metric, or some way or
measuring closeness is needed for classical clustering algorithms. Motion
patterns can be extracted from clusters.

\begin{enumerate}
\item Papers to read
\label{sec:orgdf5b87c}
That paper cited by Tiger, finding motion patterns in
video frames. By constructing a frame for each segment, the same
ideas should be applicable \cite{Kim2011Nov}

\begin{itemize}
\item \href{https://www.sciencedirect.com/science/article/abs/pii/S0031320318300621}{Structured dynamic time warping for continuous hand trajectory gesture recognition}
\item \href{https://dl.acm.org/citation.cfm?id=3140017}{A Uniform Representation for Trajectory Learning Tasks}
\end{itemize}

\item Suggested solutions
\label{sec:orgf16e770}
\begin{enumerate}
\item Constructing frames and using GPs to interpolate and synchronise
\label{sec:org7ce7afc}
After synchronised, the trajectories can be compared. This is
preferably done using a symmetric distance metric which can
then be used for clustering. \hyperref[tiger-questions]{what metrics?}

\item Dynamic Time Warping (DTW)
\label{sec:org12330aa}
Doesn't actually synchronise, but computes shortest warp path
for two trajectories. Has time and space complexity \(\mathcal{O}(NM)\)
where \(N\), $\backslash$(M)$\backslash$ are the lengths of the sequences.

\item Converting to SIT with fixed start and using sum of Euclidian distances
\label{sec:orgaad7741}
Based on \href{https://dl.acm.org/citation.cfm?id=3140017}{this paper}. With a fixed start the trajectories should
be spatially synchronised. The speed of points could be
interpolated.
\end{enumerate}
\end{enumerate}

\item {\bfseries\sffamily TODO} Clustering trajectories
\label{sec:orgb6455f1}
The number of clusters are unknown. Spectral
clustering can be used with a similarity metric, DBSCAN needs a
proper distance metric.

\begin{enumerate}
\item Papers to read
\label{sec:org6747b9d}
\begin{itemize}
\item \href{https://ieeexplore.ieee.org/abstract/document/1699726}{Comparison of Similarity Measures for Trajectory Clustering in Outdoor Surveillance Scenes}

\item \href{https://ieeexplore.ieee.org/abstract/document/994784}{Discovering similar multidimensional trajectories}
They present a new similarity measurement based on LCSS, which
is designed to be resilient to noise. Is not a proper
metric. Furthermore, they also highlight a lot of the problems
with comparing trajectories.

\item \href{../../shared/modeling-motion-patterns/vehicular-traffic-behavior-from-video.pdf}{Understanding vehicular traffic behaviour from video}
Discusses several different unsupervised
techniques. Trajectory based included, but also borrows ideas
from topic modeling in NLP.

\item \cite{Kim2011Nov}

\item \href{../../shared/clustering-trajectories/learning-and-classification-of-trajectories-in-dynamic-scenes.pdf}{Learning and Classification of Trajectories in Dynamic Scenes}
Old stuff that used HMM. But interesting approach of fitting a
Gaussian mixture model to ``points of interest'' (POI), which could be
the start and end of a trajectory in the scope of this
thesis. These POI could be used to construct frames for
synchronising trajectories.
\end{itemize}

\item Suggested solutions
\label{sec:org85fc783}
\begin{enumerate}
\item Spectral Clustering
\label{sec:org955f3ab}
Can be used with DTW/LCSS as descibed in
\cite{Zhang2006Aug}. Realistically, the evaluation would be done by hand
picking trajectories and manually asserting correct behaviour.

DTW does not guarantee that the triangle inequality holds. \hyperref[tiger-question-3]{question}

\item Inverse GP Likelihood approach
\label{sec:org9755729}
The approach used in \cite{Kim2011Nov} and in the the project
from this autumn. Based on having a probabilistic model for
each motion pattern and classifying using maximum likelihood
\end{enumerate}
\end{enumerate}
\item {\bfseries\sffamily TODO} Creating motion patterns
\label{sec:orge1b49c5}
\item {\bfseries\sffamily TODO} Classifying clusters
\label{sec:orgcd24da4}
\begin{enumerate}
\item Papers to read
\label{sec:org9ebd8c6}
One of the ones Tiger sent. Go fetch

\item Suggested solutions
\label{sec:org8d45d2f}
\begin{itemize}
\item MAP with uniform cluster prior. Requires a probabilistic
model. Assign to cluster \(k\) such that $\backslash$[
\end{itemize}
\argmax\_\{GP\(_{\text{k]}}\)\frac{1}{n}\(\sum_{\text{i=i}}^{\text{n}}\) P(GP\(_{\text{k}}\)(x\(_{\text{i}}\), y\(_{\text{i}}\)) | GP\(_{\text{k}}\))P\(_{\text{k}}\)
$\backslash$]
\end{enumerate}

\item {\bfseries\sffamily TODO} Extending to multiple stops
\label{sec:org6770255}
\begin{enumerate}
\item Papers to read
\label{sec:org6aa9936}
Theory behind simple additive model using Laplace approximation \cite{Bishop2006Aug}

\item Suggested solutions
\label{sec:org5135482}
\begin{enumerate}
\item An additive model using Laplace approximation in posterior
\label{sec:org6eb1558}
mode. Everything would be normally distributed and computable in
closed form. The posterior arrival time of segment \(k+1\) would
be \(AT_{k}\) + \(AT_{k+1}\) where \(AT_{k}\) and \(AT_{k+1}\)
are the Laplace approximations in the mode of the posteriors 
for the corresponding model \(\mathcal{M}\). For \(\mathcal{M}_{k}\) the posterior is
computed for the current state, and for \(\mathcal{M}_{k+1}\) it
is computed either for the first data point in the frame (if
frames are implemented) or for the mean value for the first data
point in the \(k+1\) segment. This would require a model for
\(P(\(\mathcal{M}_{k+1}\) | \(\mathcal{M}_{k}\)), which could be
as simple as \(\mathcal{M}_{k+1} \sim
       Dir(\alpha_{\mathcal{M}_{k}})\), where
\(\alpha_{\mathcal{M}_{k}}\) is acquired by counting and
normalising model transitions. This would be meaningless without
proper clustering, unfortunately.
\end{enumerate}
\end{enumerate}

\item {\bfseries\sffamily TODO} Extracting events from motions patterns
\label{sec:org8bb3f7d}
\begin{enumerate}
\item Papers to read
\label{sec:org6299eb1}

\item Suggested solutions
\label{sec:org18993d5}
Convolution/correlation from hand-crafted event-kernels \cite{smith1997scientist}
\end{enumerate}

\item {\bfseries\sffamily TODO} Model evaluation
\label{sec:org5d66c4f}
\begin{enumerate}
\item Arrival time prediction
\label{sec:org0528f61}
\begin{itemize}
\item Against vector field model \cite{Tran2014Jun}
\item Against TDDE19 implementation of \cite{RNNBusPredictions}
\item Use metrics MAE/MAPE
\end{itemize}

\item Motion Pattern Clustering
\label{sec:orgc6e93de}
\begin{itemize}
\item If pattern extraction works, check if they contain same patterns?
\item ?
\end{itemize}
\end{enumerate}

\item {\bfseries\sffamily TODO} Outlier detection
\label{sec:org421afc0}
\end{enumerate}

\subsubsection{{\bfseries\sffamily TODO} Write thesis plan}
\label{sec:org5321a22}
Think in terms of a divide-and-conquer approach. What problems
exist and in what order do they need to be solved?

\begin{itemize}
\item[{$\square$}] Introduction
\item[{$\square$}] Related work
\item[{$\square$}] Time plan
\end{itemize}

\subsection{Questions}
\label{sec:orgfe50123}
A place to quickly jot down questions so they are not forgotten.

\subsubsection{For Tiger}
\label{sec:org293258a}
\label{tiger-questions}
This section contain questions for Mattias Tiger, supervisor of
this thesis project.

\begin{enumerate}
\item Formulation to motivate novelty of project
\label{sec:org643388d}
\label{tiger-question-1}

\begin{enumerate}
\item Question
\label{sec:org00e0d79}
It is sort of doing arrival time prediction, but also motion pattern extraction but
also analysis of said motion patterns. Most papers found only
prove a single point, while this project builds upon several techniques.

What is the problem though? How should this be formulated in the
thesis? Is it ``Finding ways to improve public transport routes'', 
``Learning motion pattern from trajectories'', ``Motion
pattern analysis'', ``\ldots{}''?.

\item Answer
\label{sec:org2a03563}
Making a competitive GP model is interesting on it's
own. Further motivating this with the ways the motion patterns
and predictions can be used. He also said to write more than reasonable 
think on applications of the models.
\end{enumerate}


\item What distances are available for trajectories?
\label{sec:orgf83e3e8}
\label{tiger-question-3}
Spectral clustering can be done with similarity measures that are
not proper distances.
\end{enumerate}

\subsubsection{For Heintz}
\label{sec:org78a59d1}
This section contain questions for Fredrik Heintz, examiner of
this thesis project.

\subsection{Computer SSH}
\label{sec:org15a4215}
remote-und.ida.liu.se li23-[1|4]
\end{document}