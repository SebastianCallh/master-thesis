\chapter{Introduction}
\label{cha:introduction}

The introduction shall be divided into these sections:

\section{Motivation}
\label{sec:motivation}
As the years pass, more and more people move into urban areas and this
increases the importance of sustainable urban development. A greater
number of inhabitants puts higher pressure on the public
transportation systems, which makes their efficiency increasingly
important~\cite{kondepudi2014smart}. To offer a more efficient service, public traffic providers use systems
that predict arrival times of buses, trains and similar vehicles
and present this information to the inhabitants. The accuracy of these predictions are paramount, 
since many people depend on these services and erroneous predictions reflects badly on the
public traffic providers. 

Various machine learning algorithms have been applied with great
promise to predict arrival time~\cite{zheng2013urban, kim2017probabilistic, pang2018learning,
  Nguyen2018Jun}, although it is still an active research area.

\section{Aim}
\label{sec:aim}

The aim of this thesis project is to model motion patterns of public
transport vehicles using Gaussian Processes, and use these
models to make arrival time predictions with an accuracy that is
competitive to current state of the art models. Furthermore, the project aims to
extract specific events from motion patterns, such as ``emergency break'',
``standing still'', ``slow driving speed'', to see if certain events
are more common in motion patterns that do not follow the publicly
available time tables.

\section{Research questions}
\label{sec:research-questions}

This is where the research questions are described.
Formulate these as explicit questions, terminated with a
question mark. A report will usually contain several different
research questions that are somehow thematically connected.
There are usually 2-4 questions in total.

Examples of common types of research questions (simplified
and generalized):

\begin{enumerate}
\item How does technique X affect the possibility of achieving the
  effect Y?

\item How can a system (or a solution) for X be realized so
  that the effect Y is achieved?

\item What are the alternatives to
  achieving X, and which alternative gives the best effect considering
  Y and Z? (This research question is normally broken down in to 2
  separate questions.)
n
\end{enumerate}

Observe that a very specific research question almost always
leads to a better thesis report than a general research question
(it is simply much more difficult to make something good
from a general research question.)

The best way to achieve a really good and specific research
question is to conduct a thorough literature review and get
familiarized with related research and practice. This leads to
ideas and terminology which allows one to express oneself
with precision and also have something valuable to say in the
discussion chapter. And once a detailed research question
has been specified, it is much easier to establish a suitable
method and thus carry out the actual thesis work much faster
than when starting with a fairly general research question. In
the end, it usually pays off to spend some extra time in the
beginning working on the literature review. The thesis
supervisor can be of assistance in deciding when the research
question is sufficiently specific and well-grounded in related
research.

\section{Delimitations}
\label{sec:delimitations}
The data used is provided by Östgötatrafiken AB and is not publicly
available. 

This is where the main delimitations are described. For
example, this could be that one has focused the study on a
specific application domain or target user group. In the
normal case, the delimitations need not be justified.

\section{Report Outline}
