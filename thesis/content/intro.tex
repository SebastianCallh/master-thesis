\chapter{Introduction}
\label{cha:introduction}
As cities grow, efficient public transport systems
are becoming increasingly important. To offer a more efficient
service, public transport companies use systems
that predict arrival times of buses, trains and similar vehicles,
and present this information to the general public. The accuracy
and reliability of these predictions are paramount, since many 
people depend on them, and erroneous predictions reflect badly on the
public transport companies.

Machine learning algorithms have been applied with great
promise to predict arrival times~\cite{kim2017probabilistic, 
  pang2018learning, Nguyen2018Jun}, and research is still ongoing.
Modern approaches have seen heavy use of Recursive Neural Networks
(RNN) to
directly model arrival times from the current state of public
transport vehicles, an approach which has shown very good prediction accuracy
but have major drawbacks:

\begin{enumerate}
\item It does not quantify prediction uncertainties.
\item It is severely lacking in explainability.
\end{enumerate}

Knowing the uncertainty of predictions is important to avoid putting
too much faith in an ambiguous model, and being able to explain how
a model makes its predictions is equally (if not more) important,
since it allows reasoning about why a model made a certain prediction,
and about how a model would act in previously unseen scenarios. 
To address the first problem, a generative model can be used instead
of RNNs, but solving the second one and achieving a satisfying
level of explainability requires a much richer model of the data altogether.
One promising approach to this is \textit{trajectory learning}, where the goal
is to learn a \textit{motion pattern} (also known as \textit{activity
  pattern} or \textit{activity model})
which represents a cluster of individually observed
trajectories. Trajectory learning is a problem which spans several
fields, such as computer vision~\cite{Morris2008Sep, Zhang2006Aug,
  Kim2011Nov, Campo2017Aug}, pattern recognition~\cite{Tang2018Aug}, autonomous 
vehicles~\cite{Goli2018Jun}, and health informatics~\cite{Pimentel2013Sep},
and a lot of different approaches have been explored.

This thesis project aims to...
%Some of these are Hidden Markov Models
%  (HMMs)~\cite{Morris2008Sep,Suzuki2007Oct}, traditional clustering
%  using Dynamic Time Warping and Longest Common Subsequence~\cite{Tang2018Aug, Vlachos2002Feb},
%  Gaussian Processes~\cite{Tran2014Jun,
%    Pimentel2013Sep, Leysen2016Sep, Campo2017Aug, Tiger2015Jul},
%   and topic modeling~\cite{Wang2011}.
% The advantages of basing predictions of a motion pattern model is numerous, including
% detecting common characteristics, anlaysis of patterns that produce
% erroneous predictions and outlier detection.

--small pitch on what this thesis project will actually do--

\section{Aim}
\label{sec:aim}
--This is subject to change. Currently written to cover most things
talked about in the meeting from first day --

The aim of this thesis project is to model motion patterns of public
transport vehicles using Gaussian Processes (GPs), and use these
models to make arrival time predictions with an accuracy that is
competitive to current state of the art models. Furthermore, the project aims to
detect specific characteristics from motion patterns, such as ``the
driver had to emergency break'', or ``the vehicles speed was very slow'', to see if certain characteristics
are more common in motion patterns where the vehicle is too early or
too late, compared to publicly available time schedules. Finally, this
project aims to investigate how irregular motion patterns can be
detected.

\section{Research questions}
\label{sec:research-questions}

--this is subject to change. Currently written to cover most things
talked about in the meeting from first day--

\begin{enumerate}
\item How can Gaussian Processes be used to capture the motion patterns
  of trajectories?

\item How can similar trajectories be clustered
  into one representative motion pattern? 
  
\item How can specific characteristics be automatically detected in a
  motion pattern?

\item How can irregular motion patterns be detected?
\end{enumerate}

%Observe that a very specific research question almost always
%leads to a better thesis report than a general research question
%(it is simply much more difficult to make something good
%from a general research question.)

%The best way to achieve a really good and specific research
%question is to conduct a thorough literature review and get
%familiarized with related research and practice. This leads to
%ideas and terminology which allows one to express oneself
%with precision and also have something valuable to say in the
%discussion chapter. And once a detailed research question
%has been specified, it is much easier to establish a suitable
%method and thus carry out the actual thesis work much faster
%than when starting with a fairly general research question. In
%the end, it usually pays off to spend some extra time in the
%beginning working on the literature review. The thesis
%supervisor can be of assistance in deciding when the research
%question is sufficiently specific and well-grounded in related
%research.

\section{Delimitations}
\label{sec:delimitations}
%The data used is provided by Östgötatrafiken AB and is not publicly
%available. 

This is where the main delimitations are described. For
example, this could be that one has focused the study on a
specific application domain or target user group. In the
normal case, the delimitations need not be justified.

\section{Report Outline}
